\documentclass[a4paper,11pt]{article}
\usepackage[utf8]{inputenc}
\usepackage[T1]{fontenc}
\usepackage[french]{babel}
\usepackage{booktabs}
\usepackage{array}
\usepackage{siunitx}
\usepackage{geometry}
\geometry{margin=2cm}

\begin{document}

\begin{center}
\textbf{\Large Positions axiales (z) des volumes du container}
\end{center}

\vspace{0.5cm}

\noindent
\textbf{Paramètres de référence :}
\begin{itemize}
    \item Position Z du centre du container : $z_{\text{container}} = 100$ mm
    \item Hauteur intérieure du container : 12 mm
    \item Épaisseur des parois W/PETG : 2 mm
\end{itemize}

\vspace{0.5cm}

\begin{center}
\begin{tabular}{l c c c c}
\toprule
\textbf{Volume} & \textbf{Matériau} & \textbf{$z_{\min}$ (mm)} & \textbf{$z_{\max}$ (mm)} & \textbf{Épaisseur (mm)} \\
\midrule
PMMA (build-up)         & PMMA    & 90.9   & 95.9   & 5.0 \\
\midrule
PreContainerPlane       & Eau     & 95.9   & 97.9   & 2.0 \\
\midrule
Eau 1 (uniforme)        & Eau     & 95.9   & 104.9  & 9.0 \\
\midrule
Eau 2 / PostContainer   & Eau     & 104.9  & 105.9  & 1.0 \\
(anneaux concentriques) &         &        &        &     \\
\midrule
Feuille de tungstène    & W       & 105.9  & 106.0  & 0.1 \\
\midrule
Fond du container       & W/PETG  & 106.0  & 108.0  & 2.0 \\
\bottomrule
\end{tabular}
\end{center}

\vspace{0.5cm}

\noindent
\textbf{Remarques :}
\begin{itemize}
    \item Le PreContainerPlane (2 mm) chevauche le début de Eau 1 (mêmes $z_{\min}$).
    \item Les parois latérales du container (W/PETG, 2 mm d'épaisseur radiale) s'étendent de $z = 94.0$ mm à $z = 106.0$ mm.
    \item Le rayon intérieur du container est de 25 mm.
    \item Les 5 anneaux concentriques dans Eau 2 ont des rayons : 0--5, 5--10, 10--15, 15--20, 20--25 mm.
\end{itemize}

\vspace{1cm}

\begin{center}
\textbf{Schéma de l'empilement (sens des z croissants $\uparrow$)}
\end{center}

\begin{center}
\begin{tabular}{|c|c|}
\hline
$z = 108.0$ mm & \textit{Haut du fond container} \\
\hline
\multicolumn{2}{|c|}{Fond W/PETG (2 mm)} \\
\hline
$z = 106.0$ mm & \textit{Bas du fond / Haut feuille W} \\
\hline
\multicolumn{2}{|c|}{Feuille W (100 µm)} \\
\hline
$z = 105.9$ mm & \textit{Bas feuille W / Haut Eau 2} \\
\hline
\multicolumn{2}{|c|}{Eau 2 - PostContainer (1 mm) - Anneaux} \\
\hline
$z = 104.9$ mm & \textit{Bas Eau 2 / Haut Eau 1} \\
\hline
\multicolumn{2}{|c|}{Eau 1 (9 mm) - Volume uniforme} \\
\hline
$z = 95.9$ mm & \textit{Bas Eau 1 / Haut PMMA} \\
\hline
\multicolumn{2}{|c|}{PMMA (5 mm)} \\
\hline
$z = 90.9$ mm & \textit{Bas PMMA} \\
\hline
\end{tabular}
\end{center}

\end{document}
