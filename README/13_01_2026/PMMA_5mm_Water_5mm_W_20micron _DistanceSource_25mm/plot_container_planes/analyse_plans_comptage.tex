\documentclass[a4paper,11pt]{article}
\usepackage[utf8]{inputenc}
\usepackage[T1]{fontenc}
\usepackage[french]{babel}
\usepackage{graphicx}
\usepackage{booktabs}
\usepackage{array}
\usepackage{siunitx}
\usepackage{geometry}
\usepackage{amsmath}
\usepackage{amssymb}
\usepackage{caption}
\usepackage{float}
\usepackage{xcolor}
\usepackage{colortbl}
\usepackage{multirow}

\geometry{margin=2.5cm}

\sisetup{
    scientific-notation = true,
    round-mode = figures,
    round-precision = 3,
    exponent-product = \times
}

\title{\textbf{Analyse des plans de comptage PreContainer et PostContainer}\\[0.5cm]
\large Figures générées par \texttt{plot\_container\_planes.C}\\
Simulation Monte Carlo Geant4 -- Source Eu-152}
\author{}
\date{}

\begin{document}

\maketitle
\tableofcontents
\newpage

% ============================================================================
\section{Introduction}
% ============================================================================

Le script \texttt{plot\_container\_planes.C} génère 6 figures permettant d'analyser les flux de particules aux interfaces du volume d'eau. Deux plans de comptage virtuels sont définis :

\begin{itemize}
    \item \textbf{PreContainer} : Plan situé à l'entrée du volume d'eau (z = 95.9 mm), comptant les particules \textit{entrant} dans l'eau (direction +z)
    \item \textbf{PostContainer} : Plan situé à la sortie du volume d'eau (z = 104.9 mm), comptant séparément :
    \begin{itemize}
        \item Les particules \textit{transmises} (direction +z, sortant de l'eau)
        \item Les particules \textit{rétrodiffusées} (direction -z, retournant vers l'eau)
    \end{itemize}
\end{itemize}

Cette analyse permet de caractériser l'atténuation, la diffusion et la rétrodiffusion des photons et électrons dans le volume d'eau.

% ============================================================================
\section{Figure 1 : Plan PreContainer -- Particules entrant dans l'eau}
% ============================================================================

\subsection{Description}

Cette figure présente 4 histogrammes caractérisant les particules entrant dans le volume d'eau :

\begin{enumerate}
    \item Nombre de photons entrant par désintégration ($N_\gamma$ entrant)
    \item Somme des énergies des photons entrant ($\Sigma E_\gamma$)
    \item Nombre d'électrons entrant par désintégration ($N_{e^-}$ entrant)
    \item Somme des énergies des électrons entrant ($\Sigma E_{e^-}$)
\end{enumerate}

\subsection{Données observées}

\begin{table}[H]
\centering
\begin{tabular}{lccc}
\toprule
\textbf{Observable} & \textbf{Entries} & \textbf{Moyenne} & \textbf{Écart-type} \\
\midrule
$N_\gamma$ entrant & \num{2.5e7} & 2.032 & 1.332 \\
$\Sigma E_\gamma$ (keV) & \num{2.2185e7} & 1175 & 947.9 \\
$N_{e^-}$ entrant & \num{2.5e7} & 0.00653 & 0.0872 \\
$\Sigma E_{e^-}$ (keV) & \num{151560} & 480.4 & 295.1 \\
\bottomrule
\end{tabular}
\caption{Statistiques des particules entrant dans l'eau (PreContainer).}
\end{table}

\subsection{Analyse}

\begin{itemize}
    \item \textbf{Photons} : En moyenne, \textbf{2.03 photons} par désintégration atteignent l'eau, avec une énergie totale moyenne de \textbf{1175 keV}. La distribution du nombre de photons s'étend de 0 à environ 10, avec un maximum autour de 2 photons.
    
    \item \textbf{Électrons} : Très peu d'électrons primaires atteignent l'eau (moyenne = 0.0065 par événement). Ces électrons proviennent principalement de la conversion interne et des interactions dans le PMMA. Leur énergie moyenne est de \textbf{480 keV}.
    
    \item \textbf{Rapport $N_\gamma / N_{e^-}$} : Le rapport est d'environ 311, confirmant que le flux entrant est très largement dominé par les photons.
\end{itemize}

% ============================================================================
\section{Figure 2 : Photons transmis (PostContainer, +z)}
% ============================================================================

\subsection{Description}

Cette figure montre les photons qui traversent l'eau sans être absorbés et sortent par le plan PostContainer dans la direction +z (transmis).

\subsection{Données observées}

\begin{table}[H]
\centering
\begin{tabular}{lccc}
\toprule
\textbf{Observable} & \textbf{Entries} & \textbf{Moyenne} & \textbf{Écart-type} \\
\midrule
$N_\gamma$ transmis & \num{2.5e7} & 1.556 & 1.142 \\
$\Sigma E_\gamma$ transmis (keV) & \num{2.0438e7} & 986.5 & 810.2 \\
\bottomrule
\end{tabular}
\caption{Statistiques des photons transmis (PostContainer, direction +z).}
\end{table}

\subsection{Analyse}

\begin{itemize}
    \item \textbf{Transmission} : En moyenne, \textbf{1.56 photons} par événement traversent l'eau, contre 2.03 entrants. Cela correspond à un \textbf{taux de transmission de 77\%} en nombre de photons.
    
    \item \textbf{Atténuation en énergie} : L'énergie moyenne transmise est de 986.5 keV contre 1175 keV à l'entrée, soit une \textbf{atténuation de 16\%} en énergie.
    
    \item \textbf{Distribution} : Le spectre en énergie montre une structure complexe avec des pics correspondant aux raies gamma de l'Eu-152 (344, 779, 964, 1112, 1408 keV) et un continuum Compton.
\end{itemize}

% ============================================================================
\section{Figure 3 : Photons rétrodiffusés (PostContainer, -z)}
% ============================================================================

\subsection{Description}

Cette figure caractérise les photons qui, après avoir traversé partiellement l'eau, sont diffusés vers l'arrière (direction -z) et repassent par le plan PostContainer.

\subsection{Données observées}

\begin{table}[H]
\centering
\begin{tabular}{lccc}
\toprule
\textbf{Observable} & \textbf{Entries} & \textbf{Moyenne} & \textbf{Écart-type} \\
\midrule
$N_\gamma$ backscatter & \num{2.5e7} & 0.0548 & 0.245 \\
$\Sigma E_\gamma$ backscatter (keV) & \num{1274392} & 108.5 & 109.7 \\
\bottomrule
\end{tabular}
\caption{Statistiques des photons rétrodiffusés (PostContainer, direction -z).}
\end{table}

\subsection{Analyse}

\begin{itemize}
    \item \textbf{Faible rétrodiffusion} : Seulement \textbf{0.055 photons} par événement sont rétrodiffusés, ce qui correspond à environ \textbf{2.7\%} des photons entrants.
    
    \item \textbf{Énergie des photons rétrodiffusés} : L'énergie moyenne est de \textbf{108.5 keV}, nettement inférieure à l'énergie des photons entrants (1175 keV). Ceci est caractéristique de la diffusion Compton à grand angle : un photon diffusé à 180° perd une fraction importante de son énergie.
    
    \item \textbf{Origine physique} : Ces photons proviennent principalement de :
    \begin{enumerate}
        \item Diffusion Compton à grand angle dans l'eau
        \item Rétrodiffusion sur les structures après l'eau (feuille de tungstène, fond du container)
    \end{enumerate}
    
    \item \textbf{Spectre en énergie} : Le spectre montre un maximum autour de 100-200 keV, correspondant à l'énergie des photons Compton rétrodiffusés depuis les raies principales de l'Eu-152.
\end{itemize}

% ============================================================================
\section{Figure 4 : Comparaison Entrant vs Transmis (photons)}
% ============================================================================

\subsection{Description}

Cette figure superpose les distributions des photons entrant (PreContainer) et transmis (PostContainer) pour visualiser directement l'effet de l'atténuation dans l'eau.

\subsection{Données comparatives}

\begin{table}[H]
\centering
\begin{tabular}{lcccc}
\toprule
\textbf{Observable} & \textbf{PreContainer} & \textbf{PostContainer} & \textbf{Différence} & \textbf{Ratio} \\
\midrule
$\langle N_\gamma \rangle$ & 2.032 & 1.556 & -0.476 & 76.6\% \\
$\langle \Sigma E_\gamma \rangle$ (keV) & 1175 & 986.5 & -188.5 & 83.9\% \\
\bottomrule
\end{tabular}
\caption{Comparaison des photons entrant et transmis.}
\end{table}

\subsection{Analyse}

\begin{itemize}
    \item \textbf{Atténuation en nombre} : Environ \textbf{23.4\%} des photons sont absorbés ou diffusés hors du faisceau lors de la traversée de l'eau (5 mm).
    
    \item \textbf{Atténuation en énergie} : L'énergie totale transmise représente \textbf{83.9\%} de l'énergie entrante. La différence (16.1\%) correspond à l'énergie déposée dans l'eau et à l'énergie emportée par les photons diffusés.
    
    \item \textbf{Décalage des distributions} : 
    \begin{itemize}
        \item La distribution en nombre se décale vers les faibles valeurs (de 2.03 à 1.56)
        \item La distribution en énergie se décale vers les basses énergies et perd les événements de haute énergie (queue tronquée au-delà de 7000 keV)
    \end{itemize}
    
    \item \textbf{Bilan énergétique} : Sur 1175 keV entrants en moyenne :
    \begin{itemize}
        \item 986.5 keV sont transmis (83.9\%)
        \item $\sim$188.5 keV sont déposés ou diffusés (16.1\%)
    \end{itemize}
\end{itemize}

% ============================================================================
\section{Figure 5 : Électrons transmis (PostContainer, +z)}
% ============================================================================

\subsection{Description}

Cette figure caractérise les électrons qui traversent l'eau et sortent dans la direction +z.

\subsection{Données observées}

\begin{table}[H]
\centering
\begin{tabular}{lccc}
\toprule
\textbf{Observable} & \textbf{Entries} & \textbf{Moyenne} & \textbf{Écart-type} \\
\midrule
$N_{e^-}$ transmis & \num{2.5e7} & 0.00499 & 0.0733 \\
$\Sigma E_{e^-}$ transmis (keV) & \num{120479} & 474.1 & 276.5 \\
\bottomrule
\end{tabular}
\caption{Statistiques des électrons transmis (PostContainer, direction +z).}
\end{table}

\subsection{Analyse}

\begin{itemize}
    \item \textbf{Faible transmission} : Seulement \textbf{0.005 électrons} par événement traversent l'eau, ce qui est cohérent avec le faible nombre d'électrons entrants (0.0065) et l'absorption importante des électrons dans l'eau.
    
    \item \textbf{Taux de transmission électronique} : Environ \textbf{76\%} des électrons entrants traversent l'eau (0.005/0.0065). Ce taux relativement élevé s'explique par l'énergie élevée des électrons incidents (480 keV en moyenne), dont le parcours dans l'eau est de plusieurs mm.
    
    \item \textbf{Spectre en énergie} : Le spectre présente un maximum autour de \textbf{500-700 keV} et s'étend jusqu'à 2500 keV. Cette distribution reflète le spectre des électrons Compton produits par les photons de haute énergie de l'Eu-152.
    
    \item \textbf{Origine} : Ces électrons sont principalement des électrons secondaires (Compton, photoélectriques) créés près de l'interface de sortie de l'eau.
\end{itemize}

% ============================================================================
\section{Figure 6 : Électrons rétrodiffusés (PostContainer, -z)}
% ============================================================================

\subsection{Description}

Cette figure caractérise les électrons qui sont rétrodiffusés depuis les structures après l'eau (feuille de tungstène, container) et retournent vers le volume d'eau.

\subsection{Données observées}

\begin{table}[H]
\centering
\begin{tabular}{lccc}
\toprule
\textbf{Observable} & \textbf{Entries} & \textbf{Moyenne} & \textbf{Écart-type} \\
\midrule
$N_{e^-}$ backscatter & \num{2.5e7} & 0.00980 & 0.114 \\
$\Sigma E_{e^-}$ backscatter (keV) & \num{211333} & 256.3 & 273.4 \\
\bottomrule
\end{tabular}
\caption{Statistiques des électrons rétrodiffusés (PostContainer, direction -z).}
\end{table}

\subsection{Analyse}

\begin{itemize}
    \item \textbf{Rétrodiffusion significative} : Le nombre d'électrons rétrodiffusés (\textbf{0.0098}) est \textbf{deux fois supérieur} au nombre d'électrons transmis (0.005). Ceci indique une rétrodiffusion importante depuis la feuille de tungstène.
    
    \item \textbf{Coefficient de rétrodiffusion} : Le rapport backscatter/transmis est de 1.96, ce qui est caractéristique de la rétrodiffusion électronique sur un matériau à haut Z comme le tungstène (Z=74).
    
    \item \textbf{Énergie moyenne} : L'énergie moyenne des électrons rétrodiffusés est de \textbf{256 keV}, inférieure à celle des électrons transmis (474 keV). Cette perte d'énergie est due aux interactions dans la feuille de tungstène avant rétrodiffusion.
    
    \item \textbf{Spectre en énergie} : Le spectre présente un maximum à basse énergie ($<$ 200 keV) avec une décroissance rapide. La queue à haute énergie (jusqu'à 2500 keV) correspond aux électrons énergétiques rétrodiffusés élastiquement.
    
    \item \textbf{Impact dosimétrique} : Ces électrons rétrodiffusés contribuent à augmenter la dose dans la partie supérieure du volume d'eau, créant un effet de ``build-up inverse''.
\end{itemize}

% ============================================================================
\section{Synthèse et bilan des flux}
% ============================================================================

\subsection{Tableau récapitulatif}

\begin{table}[H]
\centering
\renewcommand{\arraystretch}{1.2}
\begin{tabular}{l|cc|cc}
\toprule
& \multicolumn{2}{c|}{\textbf{Photons}} & \multicolumn{2}{c}{\textbf{Électrons}} \\
\textbf{Flux} & $\langle N \rangle$ & $\langle E \rangle$ (keV) & $\langle N \rangle$ & $\langle E \rangle$ (keV) \\
\midrule
\textbf{Entrant} (Pre, +z) & 2.032 & 1175 & 0.00653 & 480 \\
\textbf{Transmis} (Post, +z) & 1.556 & 986.5 & 0.00499 & 474 \\
\textbf{Backscatter} (Post, -z) & 0.0548 & 108.5 & 0.00980 & 256 \\
\midrule
\textbf{Absorbé/diffusé} & 0.421 & 80.0 & -- & -- \\
\bottomrule
\end{tabular}
\caption{Bilan des flux de particules aux interfaces du volume d'eau.}
\end{table}

\subsection{Coefficients caractéristiques}

\begin{table}[H]
\centering
\begin{tabular}{lcc}
\toprule
\textbf{Coefficient} & \textbf{Photons} & \textbf{Électrons} \\
\midrule
Transmission ($T = N_{trans}/N_{in}$) & 76.6\% & 76.4\% \\
Rétrodiffusion ($R = N_{back}/N_{in}$) & 2.7\% & 150\%$^*$ \\
Absorption/perte ($A = 1-T-R$) & 20.7\% & -- \\
\bottomrule
\end{tabular}
\caption{Coefficients de transmission et rétrodiffusion. $^*$Le taux $>$100\% pour les électrons indique une création nette d'électrons secondaires rétrodiffusés.}
\end{table}

\subsection{Conclusions principales}

\begin{enumerate}
    \item \textbf{Photons} : L'eau de 5 mm transmet environ 77\% des photons incidents. L'atténuation (23\%) est due à l'absorption photoélectrique (raies X à 40 keV) et à la diffusion Compton. La rétrodiffusion photonique est faible (2.7\%).
    
    \item \textbf{Électrons} : Le flux électronique est dominé par les électrons secondaires. La rétrodiffusion importante depuis la feuille de tungstène (coefficient $\sim$2) crée un flux retour significatif vers l'eau.
    
    \item \textbf{Bilan énergétique} : Sur 1175 keV entrants par événement :
    \begin{itemize}
        \item 986.5 keV sont transmis (84\%)
        \item $\sim$6 keV sont rétrodiffusés (photons)
        \item $\sim$182 keV sont déposés ou perdus (16\%)
    \end{itemize}
    
    \item \textbf{Importance de la feuille de tungstène} : La feuille W (100 µm) joue un rôle crucial en rétrodiffusant les électrons vers l'eau, ce qui augmente la dose dans la partie supérieure du volume d'eau et contribue à modifier le profil de dose axial.
\end{enumerate}

% ============================================================================
\section{Implications pour la dosimétrie}
% ============================================================================

\begin{itemize}
    \item La \textbf{rétrodiffusion électronique} depuis la feuille de tungstène augmente localement la dose près de l'interface eau/tungstène.
    
    \item Les \textbf{photons rétrodiffusés} (108 keV en moyenne) sont facilement absorbés et contribuent à la dose dans la partie supérieure de l'eau.
    
    \item Le \textbf{gradient de dose} observé (Section dose par anneau) est cohérent avec ces flux : la dose est maximale au centre où les photons primaires sont les plus nombreux.
    
    \item Pour une dosimétrie précise, il est important de prendre en compte ces effets de rétrodiffusion, particulièrement pour les anneaux situés près de l'interface avec la feuille de tungstène.
\end{itemize}

\end{document}
