\documentclass[11pt,a4paper]{article}
\usepackage[utf8]{inputenc}
\usepackage[T1]{fontenc}
\usepackage[french]{babel}
\usepackage{amsmath,amssymb}
\usepackage{booktabs}
\usepackage{array}
\usepackage{geometry}
\usepackage{xcolor}
\usepackage{fancyhdr}
\usepackage{listings}
\usepackage{float}

\geometry{margin=2.5cm}

\pagestyle{fancy}
\fancyhf{}
\rhead{Simulation Monte Carlo Geant4}
\lhead{Source Eu-152}
\rfoot{Page \thepage}

\definecolor{codegreen}{rgb}{0,0.6,0}
\definecolor{codegray}{rgb}{0.5,0.5,0.5}
\definecolor{codepurple}{rgb}{0.58,0,0.82}
\definecolor{backcolour}{rgb}{0.95,0.95,0.92}

\lstdefinestyle{mystyle}{
    backgroundcolor=\color{backcolour},
    commentstyle=\color{codegreen},
    keywordstyle=\color{magenta},
    numberstyle=\tiny\color{codegray},
    basicstyle=\ttfamily\footnotesize,
    breaklines=true,
    frame=single
}
\lstset{style=mystyle}

\title{\textbf{Méthode de Génération des Événements}\\[0.5em]
\Large Source Europium-152 dans Geant4\\[0.5em]
\large Simulation Monte Carlo -- Puits Couronne}
\author{Documentation Technique}
\date{\today}

\begin{document}

\maketitle

\section{Introduction}

Ce document décrit la méthode de génération des particules primaires pour la simulation Monte Carlo d'une source d'Europium-152 ($^{152}$Eu) dans le cadre du projet ``Puits Couronne'' avec Geant4.

\subsection{Schéma de désintégration de l'Eu-152}

L'$^{152}$Eu est un radionucléide à double mode de décroissance :
\begin{itemize}
    \item \textbf{Capture électronique (EC)} : 72.1\% $\rightarrow$ $^{152}$Sm
    \item \textbf{Désintégration $\beta^-$} : 27.9\% $\rightarrow$ $^{152}$Gd
\end{itemize}

Ces deux voies de décroissance mènent à des états excités des noyaux fils, qui se désexcitent par émission de rayonnements gamma. L'$^{152}$Eu est ainsi caractérisé par un spectre gamma riche avec de nombreuses raies d'intensités variées.

\section{Philosophie de la Simulation}

\subsection{Définition d'un événement Geant4}

Dans notre simulation :
\begin{center}
\fbox{\textbf{1 événement Geant4 = 1 désintégration $^{152}$Eu}}
\end{center}

Chaque désintégration peut émettre \textbf{plusieurs photons gamma} de manière indépendante. Le nombre de photons émis par événement suit une distribution statistique déterminée par les intensités des raies.

\subsection{Relation avec l'activité}

Si la source possède une activité $A_{4\pi}$ (en Bq), alors :
\begin{equation}
\text{Nombre de désintégrations par seconde} = A_{4\pi}
\end{equation}

La correspondance entre le nombre d'événements simulés $N_{\text{evt}}$ et le temps d'irradiation $t_{\text{irr}}$ est :
\begin{equation}
t_{\text{irr}} = \frac{N_{\text{evt}}}{A_{4\pi} \times f_\Omega}
\end{equation}
où $f_\Omega$ est la fraction d'angle solide couverte par le cône d'émission.

\section{Spectre Gamma Eu-152 Implémenté}

\subsection{Raies sélectionnées}

Le spectre implémenté comprend \textbf{13 raies} (2 raies X + 11 raies gamma) avec une intensité totale de 202.78\% :

\begin{table}[H]
\centering
\caption{Spectre gamma Eu-152 implémenté (Source: NNDC/ENSDF)}
\label{tab:spectre}
\begin{tabular}{@{}clcc@{}}
\toprule
\textbf{Index} & \textbf{Énergie (keV)} & \textbf{Intensité (\%)} & \textbf{Type} \\
\midrule
0  & 39.52   & 20.80  & X (K$_\alpha$) \\
1  & 40.12   & 37.70  & X (K$_\beta$) \\
2  & 121.78  & 28.41  & $\gamma$ \\
3  & 244.70  & 7.53   & $\gamma$ \\
4  & 344.28  & 26.59  & $\gamma$ \\
5  & 411.12  & 2.24   & $\gamma$ \\
6  & 443.97  & 2.83   & $\gamma$ \\
7  & 778.90  & 12.97  & $\gamma$ \\
8  & 867.38  & 4.24   & $\gamma$ \\
9  & 964.08  & 14.63  & $\gamma$ \\
10 & 1085.87 & 10.21  & $\gamma$ \\
11 & 1112.07 & 13.64  & $\gamma$ \\
12 & 1408.01 & 21.01  & $\gamma$ \\
\midrule
\multicolumn{2}{r}{\textbf{Total :}} & \textbf{202.78} & \\
\bottomrule
\end{tabular}
\end{table}

\subsection{Nombre moyen de photons par désintégration}

Le nombre moyen de photons émis par désintégration est :
\begin{equation}
\boxed{\langle n_\gamma \rangle = \sum_{i=0}^{12} p_i = \frac{202.78}{100} \approx 2.03}
\end{equation}

Cette valeur est utilisée dans le code via la méthode \texttt{GetMeanGammasPerDecay()}.

\section{Algorithme de Génération}

\subsection{Principe : Tirages de Bernoulli indépendants}

Pour chaque événement (désintégration), l'algorithme procède comme suit :

\begin{enumerate}
    \item Pour chaque raie $i \in \{0, 1, ..., 12\}$ :
    \begin{itemize}
        \item Tirer un nombre aléatoire $r \sim \mathcal{U}(0,1)$
        \item Si $r < p_i$ (où $p_i = I_i/100$), alors la raie $i$ est émise
    \end{itemize}
    \item Pour chaque raie émise :
    \begin{itemize}
        \item Générer une direction dans le cône d'émission
        \item Créer un vertex primaire avec l'énergie $E_i$
    \end{itemize}
\end{enumerate}

\subsection{Pseudo-code}

\begin{lstlisting}[language=C++, caption={Algorithme de génération (simplifié)}]
void GeneratePrimaries(G4Event* event) {
    for (int i = 0; i < 13; i++) {
        double r = G4UniformRand();  // r in [0,1)
        
        if (r < probability[i]) {
            // Cette raie est emise
            double energy = gammaEnergies[i] * keV;
            G4ThreeVector direction = GenerateDirectionInCone();
            
            particleGun->SetParticleEnergy(energy);
            particleGun->SetParticleMomentumDirection(direction);
            particleGun->GeneratePrimaryVertex(event);
        }
    }
}
\end{lstlisting}

\subsection{Distribution du nombre de photons par événement}

Le nombre de photons $n$ émis par événement suit une loi de \textbf{Poisson composée}. En pratique, c'est la somme de 13 variables de Bernoulli indépendantes :
\begin{equation}
n = \sum_{i=0}^{12} X_i, \quad \text{où } X_i \sim \text{Bernoulli}(p_i)
\end{equation}

Les caractéristiques de cette distribution sont :
\begin{align}
\mathbb{E}[n] &= \sum_{i=0}^{12} p_i = 2.03 \\
\text{Var}(n) &= \sum_{i=0}^{12} p_i(1-p_i) \approx 1.45
\end{align}

La probabilité d'avoir un événement ``vide'' (aucun photon émis) est :
\begin{equation}
P(n=0) = \prod_{i=0}^{12} (1-p_i) \approx 0.4\%
\end{equation}

Cette probabilité est très faible car l'intensité totale est élevée (203\%).

\section{Génération de la Direction dans le Cône}

\subsection{Géométrie du cône d'émission}

La source émet dans un cône de demi-angle $\theta_{\max} = 45°$ orienté selon l'axe $+z$ :

\begin{figure}[H]
\centering
\begin{minipage}{0.6\textwidth}
\begin{align*}
&\text{Axe du cône : } \vec{u}_z = (0, 0, 1) \\
&\text{Demi-angle : } \theta_{\max} = 45° \\
&\text{Fraction d'angle solide : } f_\Omega = \frac{1 - \cos\theta_{\max}}{2} \approx 14.6\%
\end{align*}
\end{minipage}
\end{figure}

\subsection{Distribution uniforme sur la calotte sphérique}

Pour obtenir une distribution uniforme sur la surface de la calotte sphérique, on utilise :

\begin{equation}
\cos\theta = 1 - u \cdot (1 - \cos\theta_{\max}), \quad u \sim \mathcal{U}(0,1)
\end{equation}
\begin{equation}
\phi = 2\pi v, \quad v \sim \mathcal{U}(0,1)
\end{equation}

La direction est alors :
\begin{equation}
\vec{d} = (\sin\theta \cos\phi, \, \sin\theta \sin\phi, \, \cos\theta)
\end{equation}

\subsection{Code de génération}

\begin{lstlisting}[language=C++, caption={Génération de direction dans le cône}]
void GenerateDirectionInCone(G4double coneAngle,
                             G4ThreeVector& direction) {
    // cos(theta) uniforme entre cos(coneAngle) et 1
    G4double cosTheta = 1.0 - G4UniformRand() * (1.0 - cos(coneAngle));
    G4double theta = acos(cosTheta);
    
    // phi uniforme entre 0 et 2*pi
    G4double phi = G4UniformRand() * 2.0 * CLHEP::pi;
    
    // Direction cartesienne
    G4double sinTheta = sin(theta);
    direction.set(sinTheta * cos(phi),
                  sinTheta * sin(phi),
                  cosTheta);
}
\end{lstlisting}

\section{Normalisation et Calcul de Dose}

\subsection{Paramètres de la simulation}

\begin{table}[H]
\centering
\caption{Paramètres de normalisation}
\begin{tabular}{@{}lcc@{}}
\toprule
\textbf{Paramètre} & \textbf{Symbole} & \textbf{Valeur} \\
\midrule
Activité source (4$\pi$) & $A_{4\pi}$ & 42 kBq \\
Demi-angle du cône & $\theta_{\max}$ & 45° \\
Fraction d'angle solide & $f_\Omega$ & 14.6\% \\
Activité effective (cône) & $A_{\text{eff}}$ & 6.15 kBq \\
Photons moyens/désintégration & $\langle n_\gamma \rangle$ & 2.03 \\
Position source & $z_s$ & 73.5 mm \\
\bottomrule
\end{tabular}
\end{table}

\subsection{Temps d'irradiation équivalent}

Pour $N_{\text{evt}}$ événements simulés, le temps d'irradiation équivalent est :
\begin{equation}
\boxed{t_{\text{irr}} = \frac{N_{\text{evt}}}{A_{4\pi} \times f_\Omega} = \frac{N_{\text{evt}}}{42000 \times 0.146} \text{ s} = \frac{N_{\text{evt}}}{6132} \text{ s}}
\end{equation}

\subsection{Calcul du débit de dose}

Si $D_{\text{sim}}$ est la dose totale simulée (en Gy) pour $N_{\text{evt}}$ événements :
\begin{equation}
\boxed{\dot{D} = \frac{D_{\text{sim}}}{t_{\text{irr}}} = \frac{D_{\text{sim}} \times A_{4\pi} \times f_\Omega}{N_{\text{evt}}}}
\end{equation}

En nGy/s :
\begin{equation}
\dot{D} \text{ (nGy/s)} = \frac{D_{\text{sim}} \text{ (nGy)}}{N_{\text{evt}}} \times 6132
\end{equation}

\section{Résumé des Constantes dans le Code}

\begin{table}[H]
\centering
\caption{Correspondance code/physique}
\begin{tabular}{@{}lll@{}}
\toprule
\textbf{Variable C++} & \textbf{Valeur} & \textbf{Signification} \\
\midrule
\texttt{GetMeanGammasPerDecay()} & 2.03 & $\langle n_\gamma \rangle$ \\
\texttt{fMeanGammasPerDecay} & 2.03 & Idem (RunAction) \\
\texttt{fActivity4pi} & 4.2$\times 10^4$ & $A_{4\pi}$ en Bq \\
\texttt{fConeAngle} & 45° & $\theta_{\max}$ \\
\texttt{GetSolidAngleFraction()} & 0.146 & $f_\Omega$ \\
\texttt{kNbGammaLines} & 13 & Nombre de raies \\
\bottomrule
\end{tabular}
\end{table}

\section{Validation}

\subsection{Vérifications recommandées}

\begin{enumerate}
    \item \textbf{Nombre moyen de primaires} : Vérifier que $\langle n_\gamma \rangle \approx 2.03$ sur un grand nombre d'événements
    \item \textbf{Distribution angulaire} : Vérifier que $\cos\theta$ est uniforme dans $[\cos 45°, 1]$
    \item \textbf{Spectre en énergie} : Vérifier que les intensités relatives correspondent aux valeurs du tableau
    \item \textbf{Événements vides} : Vérifier que $\sim 0.4\%$ des événements n'émettent aucun photon
\end{enumerate}

\subsection{Formule de vérification}

Après $N$ événements, le nombre total de photons générés devrait être :
\begin{equation}
N_\gamma^{\text{total}} \approx N \times 2.03 \pm \sqrt{N \times 1.45}
\end{equation}

\end{document}
