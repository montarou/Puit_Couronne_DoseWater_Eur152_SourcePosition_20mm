\documentclass[a4paper,11pt]{article}
\usepackage[utf8]{inputenc}
\usepackage[T1]{fontenc}
\usepackage[french]{babel}
\usepackage{graphicx}
\usepackage{booktabs}
\usepackage{array}
\usepackage{siunitx}
\usepackage{geometry}
\usepackage{amsmath}
\usepackage{amssymb}
\usepackage{caption}
\usepackage{float}
\usepackage{xcolor}
\usepackage{colortbl}
\usepackage{multirow}

\geometry{margin=2.5cm}

\sisetup{
    scientific-notation = true,
    round-mode = figures,
    round-precision = 3,
    exponent-product = \times
}

\title{\textbf{Analyse des doses déposées par anneau d'eau}\\[0.5cm]
\large Figures générées par \texttt{analyse\_dose\_anneaux.C}\\
Simulation Monte Carlo Geant4 -- Source Eu-152}
\author{}
\date{}

\begin{document}

\maketitle

% ============================================================================
\section{Introduction}
% ============================================================================

Le script \texttt{analyse\_dose\_anneaux.C} analyse la distribution spatiale de la dose déposée dans le volume d'eau segmenté en 5 anneaux concentriques. Il génère trois figures principales :

\begin{enumerate}
    \item \textbf{histos\_dose\_par\_anneau.png} : Distribution de dose pour chaque anneau (5 panneaux)
    \item \textbf{histos\_dose\_comparaison.png} : Superposition des distributions pour comparaison directe
    \item \textbf{dose\_vs\_rayon.png} : Profil radial de la dose moyenne
\end{enumerate}

Ces figures permettent de caractériser le gradient radial de dose et la variabilité du dépôt d'énergie par événement.

% ============================================================================
\section{Figure 1 : Distributions de dose par anneau}
% ============================================================================

\subsection{Description}

Cette figure présente 5 panneaux, un pour chaque anneau d'eau. Chaque histogramme montre la distribution de la dose déposée par désintégration (en nGy) pour les événements ayant produit un dépôt d'énergie non nul dans l'anneau considéré.

\subsection{Données observées}

\begin{table}[H]
\centering
\begin{tabular}{ccccccc}
\toprule
\textbf{Anneau} & \textbf{Rayon (mm)} & \textbf{Entries} & \textbf{Mean (nGy)} & \textbf{Std Dev (nGy)} & \textbf{Dose tot. (nGy)} & \textbf{Dose max (nGy)} \\
\midrule
\rowcolor{red!15} 0 & 0--5 & 171\,227 & \num{4.27e-2} & \num{7.57e-2} & \num{7.25e3} & 0.50 \\
\rowcolor{orange!15} 1 & 5--10 & 474\,959 & \num{1.43e-2} & \num{2.53e-2} & \num{6.73e3} & 0.25 \\
\rowcolor{green!15} 2 & 10--15 & 698\,230 & \num{8.69e-3} & \num{1.53e-2} & \num{5.99e3} & 0.20 \\
\rowcolor{blue!15} 3 & 15--20 & 825\,645 & \num{6.22e-3} & \num{1.10e-2} & \num{5.12e3} & 0.15 \\
\rowcolor{violet!15} 4 & 20--25 & 884\,847 & \num{5.15e-3} & \num{8.96e-3} & \num{4.53e3} & 0.10 \\
\bottomrule
\end{tabular}
\caption{Statistiques des distributions de dose par anneau (25 millions d'événements simulés).}
\end{table}

\subsection{Analyse}

\begin{itemize}
    \item \textbf{Forme des distributions} : Toutes les distributions présentent une décroissance quasi-exponentielle, caractéristique des processus stochastiques de dépôt d'énergie. La majorité des événements déposent une faible dose, avec une queue décroissante vers les hautes doses.
    
    \item \textbf{Dose moyenne par événement déposant} : La dose moyenne diminue avec le rayon :
    \begin{itemize}
        \item Anneau 0 : \SI{4.27e-2}{\nano\gray} (le plus élevé)
        \item Anneau 4 : \SI{5.15e-3}{\nano\gray} (le plus faible)
        \item Rapport : $\sim$8.3 entre le centre et la périphérie
    \end{itemize}
    
    \item \textbf{Nombre d'événements déposants} : Le nombre d'entrées \textit{augmente} avec le rayon (171\,227 → 884\,847), car la surface des anneaux augmente ($S \propto r$). Cependant, la dose par événement diminue plus rapidement, résultant en une dose totale décroissante.
    
    \item \textbf{Écart-type} : L'écart-type diminue également avec le rayon, indiquant une distribution plus étroite pour les anneaux externes. Le rapport écart-type/moyenne reste approximativement constant ($\sim$1.7--1.8).
    
    \item \textbf{Queue de haute dose} : L'anneau central (0) présente des événements atteignant 0.5 nGy, tandis que les anneaux externes sont limités à des doses plus faibles ($<$0.2 nGy). Ces événements de haute dose correspondent à des absorptions photoélectriques complètes dans un petit volume.
\end{itemize}

% ============================================================================
\section{Figure 2 : Comparaison des distributions}
% ============================================================================

\subsection{Description}

Cette figure superpose les 5 distributions de dose sur un même graphe (échelle logarithmique en Y), permettant une comparaison directe de leur forme et de leur extension.

\subsection{Observations}

\begin{itemize}
    \item \textbf{Décalage horizontal} : Les distributions sont décalées vers les faibles doses quand le rayon augmente. L'anneau 0 (rouge) s'étend jusqu'à 0.7 nGy, tandis que l'anneau 4 (violet) est confiné en dessous de 0.1 nGy.
    
    \item \textbf{Décalage vertical} : À dose égale, le nombre d'événements augmente avec le rayon pour les faibles doses, mais diminue pour les fortes doses.
    
    \item \textbf{Croisement des courbes} : Les distributions se croisent autour de 0.02--0.03 nGy. En dessous de cette valeur, les anneaux externes dominent ; au-dessus, l'anneau central domine.
    
    \item \textbf{Pic à basse dose} : Toutes les distributions présentent un maximum autour de 0.005--0.01 nGy, correspondant aux dépôts d'énergie minimaux (électrons de faible énergie).
\end{itemize}

\subsection{Interprétation physique}

Le décalage des distributions reflète deux effets combinés :

\begin{enumerate}
    \item \textbf{Effet géométrique} : La fluence des photons diminue avec le carré de la distance à la source (pour une source ponctuelle).
    
    \item \textbf{Effet de masse} : La dose est inversement proportionnelle à la masse de l'anneau :
    \begin{equation}
        D_i = \frac{E_i}{m_i} \quad \text{avec} \quad m_i \propto (r_{ext,i}^2 - r_{int,i}^2)
    \end{equation}
    Pour un dépôt d'énergie donné, la dose est plus élevée dans les petits anneaux.
\end{enumerate}

% ============================================================================
\section{Figure 3 : Profil radial de dose}
% ============================================================================

\subsection{Description}

Cette figure présente un graphe de la dose moyenne par désintégration en fonction du rayon moyen de chaque anneau. Les points sont reliés par une ligne et annotés avec leur valeur numérique.

\subsection{Données}

\begin{table}[H]
\centering
\begin{tabular}{cccc}
\toprule
\textbf{Anneau} & \textbf{Rayon moyen (mm)} & \textbf{Dose (nGy/désint.)} & \textbf{Dose relative} \\
\midrule
\rowcolor{yellow!20} 0 & 2.5 & \num{2.69e-6} & 1.000 \\
1 & 7.5 & \num{2.40e-6} & 0.892 \\
2 & 12.5 & \num{2.40e-6} & 0.892 \\
3 & 17.5 & \num{2.05e-6} & 0.762 \\
4 & 22.5 & \num{1.81e-6} & 0.673 \\
\bottomrule
\end{tabular}
\caption{Dose moyenne par désintégration en fonction du rayon.}
\end{table}

\textbf{Note} : Ces valeurs sont calculées sur \textbf{tous} les événements (25 millions), pas seulement ceux avec dépôt. C'est pourquoi elles diffèrent des moyennes de la Figure 1.

\subsection{Analyse du gradient radial}

\begin{itemize}
    \item \textbf{Décroissance monotone} : La dose décroît de façon monotone avec le rayon, de \SI{2.69e-6}{\nano\gray} au centre à \SI{1.81e-6}{\nano\gray} à la périphérie.
    
    \item \textbf{Taux de décroissance} : La dose diminue de \textbf{32.7\%} entre le centre et la périphérie (rapport 1.49).
    
    \item \textbf{Forme du profil} : Le profil est approximativement linéaire en fonction du rayon :
    \begin{equation}
        D(r) \approx D_0 - \alpha \cdot r
    \end{equation}
    avec $D_0 \approx \SI{2.9e-6}{\nano\gray}$ et $\alpha \approx \SI{4.4e-8}{\nano\gray/\milli\metre}$.
    
    \item \textbf{Erreurs statistiques} : Les erreurs statistiques sont très faibles ($\sim$\SI{e-8}{\nano\gray}), ce qui confirme la bonne convergence de la simulation avec 25 millions d'événements.
\end{itemize}

\subsection{Modélisation du gradient}

Le gradient radial peut s'expliquer par la combinaison de :

\begin{enumerate}
    \item \textbf{Divergence géométrique} : Pour une source ponctuelle, la fluence varie en $1/r^2$. Cependant, ici la géométrie est plus complexe (source étendue, parois réfléchissantes).
    
    \item \textbf{Atténuation radiale} : Les photons traversant radialement l'eau subissent une atténuation supplémentaire.
    
    \item \textbf{Contribution des électrons secondaires} : Les électrons Compton ont un parcours limité ($\sim$quelques mm) et déposent préférentiellement leur énergie près de leur point de création.
\end{enumerate}

Un ajustement empirique donne :
\begin{equation}
D(r) = D_0 \cdot \left(1 - \beta \cdot r\right) \quad \text{avec} \quad \beta \approx 0.016 \text{ mm}^{-1}
\end{equation}

% ============================================================================
\section{Implications pour la dosimétrie}
% ============================================================================

\subsection{Non-uniformité de dose}

Le rapport de dose entre le centre et la périphérie est de \textbf{1.49}, ce qui représente une non-uniformité de \textbf{49\%}. Cette valeur est significative pour les applications dosimétriques nécessitant une irradiation homogène.

\subsection{Facteurs de correction par anneau}

Pour obtenir une dose uniforme, il faudrait appliquer des facteurs de correction :

\begin{table}[H]
\centering
\begin{tabular}{ccc}
\toprule
\textbf{Anneau} & \textbf{Dose relative} & \textbf{Facteur de correction} \\
\midrule
0 & 1.000 & 1.000 \\
1 & 0.892 & 1.121 \\
2 & 0.892 & 1.121 \\
3 & 0.762 & 1.312 \\
4 & 0.673 & 1.486 \\
\bottomrule
\end{tabular}
\caption{Facteurs de correction pour uniformiser la dose (référence : anneau central).}
\end{table}

\subsection{Optimisation possible}

Pour réduire le gradient de dose, on pourrait :
\begin{itemize}
    \item Augmenter la distance source-échantillon (meilleure uniformité mais réduction du débit)
    \item Utiliser un diffuseur pour homogénéiser le faisceau
    \item Faire tourner l'échantillon pendant l'irradiation
\end{itemize}

% ============================================================================
\section{Synthèse}
% ============================================================================

\begin{table}[H]
\centering
\renewcommand{\arraystretch}{1.2}
\begin{tabular}{ll}
\toprule
\textbf{Paramètre} & \textbf{Valeur} \\
\midrule
Nombre d'événements simulés & \num{2.5e9} (25 millions $\times$ 100) \\
Dose moyenne (anneau central) & \SI{2.69e-6}{\nano\gray/désint.} \\
Dose moyenne (anneau externe) & \SI{1.81e-6}{\nano\gray/désint.} \\
Gradient radial & -32.7\% (centre → périphérie) \\
Rapport max/min & 1.49 \\
Erreur statistique relative & $<$0.1\% \\
\bottomrule
\end{tabular}
\caption{Synthèse des résultats de l'analyse dosimétrique par anneau.}
\end{table}

\subsection{Points clés}

\begin{enumerate}
    \item Les distributions de dose par événement suivent une \textbf{décroissance quasi-exponentielle}, caractéristique des processus stochastiques d'interaction rayonnement-matière.
    
    \item La dose moyenne par désintégration \textbf{décroît linéairement} avec le rayon, avec une réduction de 33\% entre le centre et la périphérie.
    
    \item Le \textbf{nombre d'événements déposants augmente} avec le rayon (effet de surface), mais la \textbf{dose par événement diminue} plus rapidement (effet géométrique + masse).
    
    \item Les \textbf{erreurs statistiques} sont négligeables ($\sim$\SI{e-8}{\nano\gray}), confirmant la convergence de la simulation.
    
    \item Pour une dosimétrie précise, il est nécessaire de prendre en compte le \textbf{gradient radial de dose} et d'appliquer des corrections appropriées selon l'anneau de mesure.
\end{enumerate}

\end{document}
