\documentclass[a4paper,11pt]{article}
\usepackage[utf8]{inputenc}
\usepackage[T1]{fontenc}
\usepackage[french]{babel}
\usepackage{graphicx}
\usepackage{booktabs}
\usepackage{array}
\usepackage{siunitx}
\usepackage{geometry}
\usepackage{amsmath}
\usepackage{amssymb}
\usepackage{caption}
\usepackage{float}
\usepackage{xcolor}
\usepackage{colortbl}
\usepackage{diagbox}

\geometry{margin=2.5cm}

\sisetup{
    scientific-notation = true,
    round-mode = figures,
    round-precision = 3,
    exponent-product = \times
}

\title{\textbf{Analyse dosimétrique par anneau}\\[0.5cm]
\large Simulation Monte Carlo Geant4 -- Source Eu-152 (42 kBq)\\
Configuration sans filtre -- 25 millions d'événements}
\author{}
\date{}

\begin{document}

\maketitle

% ============================================================================
\section{Paramètres de la simulation}
% ============================================================================

\begin{table}[H]
\centering
\begin{tabular}{ll}
\toprule
\textbf{Paramètre} & \textbf{Valeur} \\
\midrule
Nombre d'événements simulés & \num{25000000} \\
Gammas primaires générés & \num{50701352} \\
Gammas entrant dans l'eau & \num{48288275} \\
Gammas absorbés dans l'eau & \num{353724} \\
Électrons secondaires dans l'eau & \num{238852} \\
Énergie totale déposée dans l'eau & \SI{328.47}{\mega\eV} \\
Activité de la source (supposée) & \SI{42}{\kilo\becquerel} \\
Configuration & Sans filtre W/PETG \\
\bottomrule
\end{tabular}
\caption{Paramètres et statistiques de la simulation.}
\end{table}

% ============================================================================
\section{Géométrie des anneaux d'eau}
% ============================================================================

Le volume d'eau est segmenté en 5 anneaux concentriques de 5 mm de largeur chacun. Les caractéristiques géométriques sont :

\begin{table}[H]
\centering
\begin{tabular}{ccccc}
\toprule
\textbf{Anneau} & \textbf{$r_{int}$ (mm)} & \textbf{$r_{ext}$ (mm)} & \textbf{Volume (cm$^3$)} & \textbf{Masse (g)} \\
\midrule
0 & 0 & 5 & 0.3927 & 0.3927 \\
1 & 5 & 10 & 1.1781 & 1.1781 \\
2 & 10 & 15 & 1.9635 & 1.9635 \\
3 & 15 & 20 & 2.7489 & 2.7489 \\
4 & 20 & 25 & 3.5343 & 3.5343 \\
\midrule
\textbf{Total} & 0 & 25 & 9.8175 & 9.8175 \\
\bottomrule
\end{tabular}
\caption{Caractéristiques géométriques des anneaux d'eau (épaisseur 5 mm, $\rho = 1$ g/cm$^3$).}
\end{table}

Le volume de chaque anneau est calculé par :
\begin{equation}
V_i = \pi \left( r_{ext,i}^2 - r_{int,i}^2 \right) \times h
\end{equation}
où $h = 5$ mm est l'épaisseur de la couche d'eau.

% ============================================================================
\section{Analyse des doses déposées par anneau}
% ============================================================================

\subsection{Résultats de la simulation}

\begin{table}[H]
\centering
\begin{tabular}{cccccc}
\toprule
\textbf{Anneau} & \textbf{Rayon (mm)} & \textbf{Masse (g)} & \textbf{Énergie (MeV)} & \textbf{Dose (nGy/evt)} & \textbf{Dose relative} \\
\midrule
\rowcolor{yellow!30} 0 & 0 -- 5 & 0.3927 & \num{1.776e4} & \num{2.899e-4} & 1.000 \\
1 & 5 -- 10 & 1.1781 & \num{4.946e4} & \num{2.690e-4} & 0.928 \\
2 & 10 -- 15 & 1.9635 & \num{7.343e4} & \num{2.397e-4} & 0.827 \\
3 & 15 -- 20 & 2.7489 & \num{8.780e4} & \num{2.047e-4} & 0.706 \\
4 & 20 -- 25 & 3.5343 & \num{1.000e5} & \num{1.814e-4} & 0.626 \\
\bottomrule
\end{tabular}
\caption{Énergie déposée et dose par anneau. L'anneau 0 (central) est surligné.}
\label{tab:dose_anneau}
\end{table}

\subsection{Interprétation physique}

\subsubsection{Gradient radial de dose}

On observe une décroissance monotone de la dose avec le rayon :
\begin{itemize}
    \item \textbf{Anneau central (0)} : $D_0 = \SI{2.899e-4}{\nano\gray/event}$ (dose maximale)
    \item \textbf{Anneau externe (4)} : $D_4 = \SI{1.814e-4}{\nano\gray/event}$ (dose minimale)
    \item \textbf{Rapport} : $D_4/D_0 = 0.626$ soit une réduction de 37.4\%
\end{itemize}

Ce gradient s'explique par :
\begin{enumerate}
    \item La \textbf{divergence géométrique} du faisceau depuis la source ponctuelle
    \item L'\textbf{atténuation radiale} des photons diffusés
    \item Le \textbf{parcours limité} des électrons secondaires ($\sim$ quelques mm dans l'eau)
\end{enumerate}

\subsubsection{Énergie déposée vs. masse}

Bien que l'énergie totale déposée augmente avec le rayon (car la surface augmente), la dose \textit{par unité de masse} diminue :

\begin{equation}
D_i = \frac{E_i}{m_i} \quad \text{avec} \quad m_i \propto \left( r_{ext,i}^2 - r_{int,i}^2 \right)
\end{equation}

L'anneau 4 reçoit la plus grande énergie totale (\SI{1.0e5}{\mega\eV}) mais a la dose la plus faible car sa masse est la plus grande (\SI{3.53}{\gram}).

% ============================================================================
\section{Calcul des temps d'irradiation}
% ============================================================================

\subsection{Méthodologie}

Pour une source d'activité $A$ (en Bq), le débit de dose dans l'anneau $i$ est :
\begin{equation}
\dot{D}_i = D_i \times A
\end{equation}
où $D_i$ est la dose par événement (nGy/evt) et $A$ représente le nombre de désintégrations par seconde.

Le temps nécessaire pour atteindre une dose cible $D_{cible}$ est :
\begin{equation}
t = \frac{D_{cible}}{\dot{D}_i} = \frac{D_{cible}}{D_i \times A}
\end{equation}

\subsection{Application à l'anneau central (Anneau 0)}

Pour l'anneau central avec une source Eu-152 de \SI{42}{\kilo\becquerel} :

\begin{align}
D_0 &= \SI{2.899e-4}{\nano\gray/event} = \SI{2.899e-13}{\gray/event} \\
A &= \SI{42000}{\becquerel} = \SI{42000}{event/\second} \\
\dot{D}_0 &= \SI{2.899e-13}{\gray/event} \times \SI{42000}{event/\second} \\
\dot{D}_0 &= \SI{1.218e-8}{\gray/\second} = \SI{1.218e-6}{c\gray/\second}
\end{align}

En unités plus pratiques :
\begin{equation}
\boxed{\dot{D}_0 = \SI{1.052}{m\gray/jour} = \SI{0.1052}{c\gray/jour}}
\end{equation}

\subsection{Temps pour atteindre les doses cibles}

\begin{table}[H]
\centering
\renewcommand{\arraystretch}{1.3}
\begin{tabular}{ccccc}
\toprule
\textbf{Dose cible} & \textbf{Dose (Gy)} & \textbf{Temps (s)} & \textbf{Temps (jours)} & \textbf{Temps (mois)} \\
\midrule
\rowcolor{green!20} \SI{10}{c\gray} & 0.1 & \num{8.21e6} & \textbf{95.0} & 3.17 \\
\rowcolor{orange!20} \SI{20}{c\gray} & 0.2 & \num{1.64e7} & \textbf{190.1} & 6.34 \\
\rowcolor{red!20} \SI{50}{c\gray} & 0.5 & \num{4.11e7} & \textbf{475.2} & 15.8 \\
\bottomrule
\end{tabular}
\caption{Temps d'irradiation pour atteindre les doses cibles dans l'anneau central (source Eu-152, \SI{42}{\kilo\becquerel}).}
\label{tab:temps_irradiation}
\end{table}

\subsection{Formules de calcul détaillées}

Pour \textbf{10 cGy} :
\begin{equation}
t_{10} = \frac{\SI{0.1}{\gray}}{\SI{1.218e-8}{\gray/\second}} = \SI{8.21e6}{\second} = \SI{95.0}{jours} \approx \textbf{3.2 mois}
\end{equation}

Pour \textbf{20 cGy} :
\begin{equation}
t_{20} = \frac{\SI{0.2}{\gray}}{\SI{1.218e-8}{\gray/\second}} = \SI{1.64e7}{\second} = \SI{190.1}{jours} \approx \textbf{6.3 mois}
\end{equation}

Pour \textbf{50 cGy} :
\begin{equation}
t_{50} = \frac{\SI{0.5}{\gray}}{\SI{1.218e-8}{\gray/\second}} = \SI{4.11e7}{\second} = \SI{475.2}{jours} \approx \textbf{1.3 ans}
\end{equation}

% ============================================================================
\section{Analyse comparative entre anneaux}
% ============================================================================

\subsection{Débits de dose par anneau}

\begin{table}[H]
\centering
\begin{tabular}{ccccc}
\toprule
\textbf{Anneau} & \textbf{Dose (nGy/evt)} & \textbf{Débit (nGy/s)} & \textbf{Débit (mGy/jour)} & \textbf{Temps pour 10 cGy (jours)} \\
\midrule
\rowcolor{yellow!30} 0 & \num{2.899e-4} & 12.18 & 1.052 & \textbf{95.0} \\
1 & \num{2.690e-4} & 11.30 & 0.976 & 102.4 \\
2 & \num{2.397e-4} & 10.07 & 0.870 & 114.9 \\
3 & \num{2.047e-4} & 8.60 & 0.743 & 134.6 \\
4 & \num{1.814e-4} & 7.62 & 0.658 & 151.9 \\
\bottomrule
\end{tabular}
\caption{Débits de dose et temps d'irradiation pour chaque anneau (source \SI{42}{\kilo\becquerel}).}
\end{table}

\subsection{Uniformité de dose}

Le rapport entre la dose maximale (anneau 0) et minimale (anneau 4) est :
\begin{equation}
\frac{D_{max}}{D_{min}} = \frac{D_0}{D_4} = \frac{2.899}{1.814} = 1.60
\end{equation}

Cette non-uniformité de 60\% est significative et doit être prise en compte pour les applications dosimétriques nécessitant une irradiation homogène.

% ============================================================================
\section{Taux d'absorption par raie gamma}
% ============================================================================

L'analyse par raie gamma montre que l'absorption est dominée par les raies X de basse énergie :

\begin{table}[H]
\centering
\begin{tabular}{cccccc}
\toprule
\textbf{Raie} & \textbf{Énergie (keV)} & \textbf{Émis} & \textbf{Entré eau} & \textbf{Absorbé} & \textbf{Taux (\%)} \\
\midrule
\rowcolor{blue!15} 0 & 39.5 (X) & \num{5201331} & \num{4693419} & \num{127786} & \textbf{2.72} \\
\rowcolor{blue!15} 1 & 40.1 (X) & \num{9423209} & \num{8517840} & \num{220098} & \textbf{2.58} \\
2 & 121.8 & \num{7101317} & \num{6740725} & \num{5097} & 0.08 \\
3 & 244.7 & \num{1882686} & \num{1811268} & \num{209} & 0.01 \\
4 & 344.3 & \num{6650954} & \num{6439917} & \num{254} & $<$0.01 \\
\bottomrule
\end{tabular}
\caption{Taux d'absorption dans l'eau par raie gamma (5 premières raies). Les raies X sont surlignées.}
\end{table}

\textbf{Observation clé} : Les raies X à $\sim$40 keV contribuent à plus de 98\% des absorptions dans l'eau, avec un taux d'absorption d'environ 2.7\%.

% ============================================================================
\section{Synthèse et conclusions}
% ============================================================================

\subsection{Résultats principaux}

\begin{enumerate}
    \item \textbf{Dose dans l'anneau central} : $D_0 = \SI{2.899e-4}{\nano\gray}$ par désintégration
    
    \item \textbf{Débit de dose} (source \SI{42}{\kilo\becquerel}) : $\dot{D}_0 = \SI{1.05}{m\gray/jour}$
    
    \item \textbf{Temps d'irradiation} pour l'anneau central :
    \begin{itemize}
        \item 10 cGy : \textbf{95 jours} ($\sim$ 3.2 mois)
        \item 20 cGy : \textbf{190 jours} ($\sim$ 6.3 mois)  
        \item 50 cGy : \textbf{475 jours} ($\sim$ 1.3 ans)
    \end{itemize}
    
    \item \textbf{Gradient radial} : Réduction de 37\% de la dose entre le centre et la périphérie
\end{enumerate}

\subsection{Implications pratiques}

\begin{itemize}
    \item Les temps d'irradiation sont très longs en raison de la faible activité de la source (\SI{42}{\kilo\becquerel})
    \item Pour réduire les temps d'un facteur 10, il faudrait une source de \SI{420}{\kilo\becquerel}
    \item L'absorption est dominée par les raies X de basse énergie ($\sim$40 keV)
    \item Le gradient radial de dose nécessite une calibration spécifique pour chaque anneau
\end{itemize}

\subsection{Formule générale}

Pour une activité $A$ quelconque (en kBq), le temps d'irradiation pour atteindre une dose $D$ (en cGy) dans l'anneau central est :
\begin{equation}
\boxed{t \text{ (jours)} = \frac{D \text{ (cGy)}}{0.1052} \times \frac{42}{A \text{ (kBq)}} = \frac{399.2 \times D}{A}}
\end{equation}

\vspace{1cm}

\begin{table}[H]
\centering
\renewcommand{\arraystretch}{1.2}
\begin{tabular}{c|ccc}
\toprule
\diagbox{\textbf{Activité}}{\textbf{Dose}} & \textbf{10 cGy} & \textbf{20 cGy} & \textbf{50 cGy} \\
\midrule
\SI{42}{\kilo\becquerel} & 95 jours & 190 jours & 475 jours \\
\SI{100}{\kilo\becquerel} & 40 jours & 80 jours & 200 jours \\
\SI{420}{\kilo\becquerel} & 9.5 jours & 19 jours & 47.5 jours \\
\SI{1}{\mega\becquerel} & 4 jours & 8 jours & 20 jours \\
\bottomrule
\end{tabular}
\caption{Temps d'irradiation en fonction de l'activité de la source pour différentes doses cibles.}
\end{table}

\end{document}
